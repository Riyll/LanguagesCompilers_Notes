\chapter{Lexical Analysis}

\section{Reminder about compilation steps}
\begin{enumerate}
    \item \textbf{Scanner/Lexer}: This takes a Source program and breaks it into tokens.
    \item \textbf{Parser}: This takes a stream of tokens and builds a parse tree.
    \item \textbf{Semantic Analyzer}: Takes cares of the annotations and type checking. This generated an annotated tree.\\
    \hline
    From here on this low level optimization
    \hline
\item \textbf{Source code optimizer}: Converts the Annotated tree into intermediate code.
\item \textbf{Code Generator}: Generates the target code.
\item \textbf{Target Code Optimizer}: Optimizes the target code.
\end{enumerate}

It is important to note that a Lexical Analyzer is a synonym of a Lexer and a Scanner. The terms are used interchangeably.

\section{Lexical Analysis}
Lexical analysis is the process of converting a sequence of characters into a sequence of tokens. The tokens are the smallest meaningful units of a programming language. For example, in C, the tokens are identifiers, keywords, operators, and so on. The lexical analyzer is also known as a scanner or a lexer.\\
This removes the content-free characters such as spaces, tabs, and newlines. It also removes comments. 
\\ It detects lexical errors, such as badly-formed literals, and illegal characters.\\
This is a pattern matching process: It acts as front-eend for the parser, and is an implementation of a finite state automaton.\\
A substring of the source program that belongs together is called a lexeme, and a lexeme is matched against a pattern, which is associated with a lexical category called a \textbf{token}.\\

You can define a lexical analyzer via a list of pairs:
\begin{itemize}
    \item A regular Expression: describes a token pattern that is matched against the input to extract lexemes.
    \item An action: a piece of code, parameterized by the matching lexeme, that returns a (token, attribute) pair.
\end{itemize}   

\section{Regular Expressions}
These are a language with its own syntax and semantics. They are used to describe patterns in strings.\\
Certain characters and character combinations denote specific operations, impose certain constraints, or modify default operator precedence.
\textbf{Meta languages}: They describe a particular family of languages called the \textbf{regular languages}.\\

\subsection{Important Symbols}
The empty set: $\emptyset$\\
The empty string: $\epsilon$\\
Alternation: $|$\\
Repetition: $*$: 0 or more times, $+$: 1 or more times, $?$: 0 or 1 times\\
Optionality: $?$: a$?$ describes the set of strings that are either a or $\epsilon$\\
A kleene cross: $+$: a$+$ describes the set of strings that are either a or aa or aaa or ...\\
Wildcard: $.$: matches any single character\\
Classes: $[a-z]$\\
Escape: $\backslash$\\

\subsection{Precedence}
The precedence of the operators is as follows:
\begin{enumerate}
    \item $*$, $+$, $?$, $^*$, $^+$, $^?$, $^-$
    \item Concatenation
    \item Alternation
\end{enumerate} 
\newpage

\section{Finite State Automata}
A finite state automaton (FSA) is a machine that has a finite number of states and a finite number of inputs.\\
It is a model of computation that can be used to recognize languages.\\
It is a 5-tuple: $M = (Q, \Sigma, \delta, q_0, F)$\\
Where:
\begin{itemize}
    \item $Q$ is a finite set of states.
    \item $\Sigma$ is a finite set of input symbols.
    \item $\delta: Q \times \Sigma \rightarrow Q$ is the transition function.
    \item $q_0 \in Q$ is the initial state.
    \item $F \subseteq Q$ is the set of final states. (accepting states)
\end{itemize}

\subsection{Deterministic Finite State Automata}
A deterministic finite state automaton (DFA) is a finite state automaton that has the following properties:
\begin{enumerate}
    \item For each state $q \in Q$ and each input symbol $a \in \Sigma$, there is at most one transition from $q$ on $a$.
    \item The transition function $\delta$ is total: for each state $q \in Q$ and each input symbol $a \in \Sigma$, there is a transition from $q$ on $a$.
\end{enumerate}

\subsection{Non-Deterministic Finite State Automata}
A non-deterministic finite state automaton (NFA) is a finite state automaton that is not necessarily deterministic.\\
an NFA accepts a string when there is a computation of the NFA that accepts the string. \\
There is a \textbf{Computation} when all the input is consumed and the automaton is in an accepting state.\\
An NFA rejects a string when there is no computation of the NFA that accepts the string.\\

Empty transitions are allowed in an NFA, but not in DFAs.


\subsection{DFA vs NFA}

\begin{itemize}
    \item NFAs are usually smaller but take more time to process, since they have to backtrack if they tak the wrong path, or even hang.
    \item Every NFA can be transformed into a DFA, but the resulting DFA may be much larger.
    \item Every NFA with lambda transitions can be transformed into an equivalent NFA without lambda transitions.
\end{itemize}

\section{Lexical Analyzer Implementation}
You might design a finiet state automaton that d escribes the expected input patterns and writes a program that implements the automaton.\\
This is a tedious process, and it is error-prone.\\
\noindent\hfil\rule{0.5\textwidth}{.4pt}\hfil
\\
You might design a FSA that describes the expected input patterns and hand-construct a table-driven implementation of the FSA.
\\
\noindent\hfil\rule{0.5\textwidth}{.4pt}\hfil
\\
You might write a formal description of he expected input patterns using REs and use a software tool that constructs table-driven lexical analyzers from such a description.

\section{State Diagram Design}
This is one way of representing an FSA.\\
A naïve state diagram would have a transition from every state on every character in the source language. But it would be very large!\\ 
In many cases, transitions can be combined to simplify the state diagram:
\begin{itemize}
     \item When recognizing an identifier, all uppercase and lowercase letters are equivalent. So, you can combine all the transitions on the letters into a single transition on a class of letters.
    \item When recognizing a number, all digits are equivalent. So, you can combine all the transitions on the digits into a single transition on a class of digits.
\end{itemize}

an example of this code in C would look like:

\begin{cppcode}
    int LexAnalyzer() {
        getChar();
        if (isLetter(nextChar)) {
            addChar();
            getChar();
            while (isLetter(nextChar) || isDigit(nextChar)) {
                addChar();
                getChar();
            }
            return lookup(lexeme);
        }
        else if (isDigit(nextChar)) {
            addChar();
            getChar();
            while (isDigit(nextChar)) {
                addChar();
                getChar();
            }
            return INT_LIT;
        }
    }
\end{cppcode}

You could try to match all reserved words explicitly in the FSA, but this would be tedious and error-prone.\\
It is better to treat all IDs the same but look them up in the \textbf{Symbol Table} after the token is created to identify which are reserved words.\\ 

A lexical analyzer using token parsing with FSA logic implementation algorithm would look like:

\begin{commandshell}
    state =S // S is the start state
    repeat {
        k = next character from the input
        if k == EOF                     // the end of input
            if state is a final state
                then accept
                else reject
        state = transition(state, k)    // transition to a new state
        if state == empty
            then reject                 // got stuck
        }
\end{commandshell}

\section{On Lexical Errors}
Lexical errors are not syntax errors,  they occur only if a character not recognized by a programming language.\\
This usually means that you have a malformed lexeme \textbf{According to the rules of the language}.\\

\section{Practice Problems}
This section will be added later on.
